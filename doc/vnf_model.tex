
The VNF catalogue supported by VeriGraph is composed of several network function models, which are: End-host, Mail Server/Client, Web Client/Server, Anti-spam, NAT, Web Cache, ACL firewall and Learning Firewall. Here we describe the formulas that model the functional behaviour of each function in catalog.


\paragraph{End-host model}
An end-host is a network node which sends packets towards a destination and receives packets from a source. The sent packets must satisfy some conditions (Formula~\ref{end-host_constr1}): \textit{(i)} the end-host address is the source address; \textit{(ii)} \textit{origin} is the end-host itself; \textit{(iii)} \textit{origin\_body} and \textit{body}  must be equal. The received packet (Formula~\ref{end-host_constr2}) must have the end-host address as destination. 
\begin{figure}[h]
	{\footnotesize
		\begin{subequations}
			%Antispam model:
			\begin{align}
				\begin{split}
					\label{end-host_constr1}
					(send(end\_host, n_{0}, p_{0}, t_{0}) &) \implies (nodeHasAddr(end\_host, p_{0}.src) \wedge \\
					& p_{0}.origin == end\_host \wedge p_{0}.origin\_body == p_{0}.body) \\
					& \forall (n_{0}, p_{0}, t_{0})
				\end{split} \\
				\begin{split}
					\label{end-host_constr2}
					(recv(n_{0}, end\_host, p_{0}, t_{0}) &) \implies (nodeHasAddr(end\_host, p_{0}.dest)),\\
					& \forall (n_{0}, p_{0}, t_{0})
				\end{split}
			\end{align}
		\end{subequations}
	}%
	\caption{End-host model.}
\end{figure}
\\
At this version of VeriGraph, the tool does not support end-host configurations, as well as the other end-host-based models (i.e., servers and clients). In other words, we cannot specify which traffic flow an end-host sends, without changing its model (e.g., a client can generate packet with specific port number, destination address etc.). It could be useful to extend VeriGraph to support end-host configurations.

\paragraph{Mail Server model} A mail server is a complex form of end-host. In fact, this kind of server can generate only \textit{POP3\_RESPONSE} or \textit{SMTP\_RESPONSE} messages addressed to a mail client (Formula~\ref{mail-server_constr1}), but the type of the response depends on the previous messages. In particular, a \textit{POP3\_RESPONSE} packet is sent only if previously a \textit{POP3\_REQUEST} was received (Formula~\ref{mail-server_constr2}) and similarly for the \textit{SMTP\_RESPONSE} packets (Formula~\ref{mail-server_constr3}). A mail server is also modelled for receiving just \textit{POP3\_REQUEST} or \textit{SMTP\_REQUEST} messages (Formula~\ref{mail-server_constr4}). 
\begin{figure}[h]
	{\footnotesize
		\begin{subequations}
			%MailServer model:
			\begin{align}
				\begin{split}
					\label{mail-server_constr1}
					(se& nd(mail\_server , n_{0}, p_{0}, t_{0})) \implies (nodeHasAddr(mail\_server, p_{0}.src) \wedge \\
					& p_{0}.origin = mail\_server \wedge p_{0}.origin\_body = p_{0}.body \wedge \\
					& (p_{0}.proto = POP3\_RESPONSE \vee p_{0}.proto = SMTP\_RESPONSE) \wedge \\
					& p_{0}.dst = ip\_mail\_client \wedge p_{0}.emailFrom = mail\_server), \\
					& \forall (n_{0}, p_{0}, t_{0})
				\end{split} \\
%			\end{align}
%		\end{subequations}
%	}
%\end{figure}
%\begin{figure}
%	{\footnotesize
%		\begin{subequations} 
%		\begin{align}
				\begin{split}
					\label{mail-server_constr2}
					(se& nd(mail\_server , n_{0}, p_{0}, t_{0}) \wedge p_{0}.proto = POP3\_RESPONSE) \implies \\
					& \exists (p_{1}, t_{1}) | recv(n_{0}, mail\_server, p_{1}, t_{1})  \wedge p_{1}.proto = POP3\_REQUEST),\\
					& \forall (n_{0}, p_{0}, t_{0})
				\end{split} \\
				\begin{split}
					\label{mail-server_constr3}
					(se& nd(mail\_server , n_{0}, p_{0}, t_{0}) \wedge p_{0}.proto = SMTP\_RESPONSE) \implies \\
					& \exists (p_{1}, t_{1}) | recv(n_{0}, mail\_server, p_{1}, t_{1}) \wedge  p_{1}.proto = SMTP\_REQUEST),\\
					& \forall (n_{0}, p_{0}, t_{0})
				\end{split} \\
				\begin{split}
					\label{mail-server_constr4}
					(re& cv(n_{0}, mail\_server, p_{0}, t_{0})) \implies  (nodeHasAddr(mail\_server, p_{0}.dst) \\
					& (p_{0}.proto = POP3\_REQUEST \vee p_{0}.proto = SMTP\_REQUEST)), \\
					& \forall (n_{0}, p_{0}, t_{0})
				\end{split}
			\end{align}
		\end{subequations}
	}%
	\caption{Mail Server model.}
	\label{mail-server_model}
\end{figure}

\paragraph{Mail client model} A mail client is a particular kind of end-host (Formulas~\ref{mail-client_constr1} and~\ref{mail-client_constr2}). This node is modelled so that it can send \textit{POP3\_REQUEST} or \textit{SMTP\_REQUEST} messages only. Those messages must have a mail server address as destination (\textit{$p_{0}.dst = ip\_mail\_server$}) and the \textit{emailFrom} field must indicate the mail client node itself (Formula~\ref{mail-client_constr1}).
\begin{figure}[h]
	{\footnotesize
		\begin{subequations}
			%MailClient model:
			\begin{align}
				\begin{split}
					\label{mail-client_constr1}
					(se& nd(mail\_client , n_{0}, p_{0}, t_{0})) \implies (nodeHasAddr(mail\_client, p_{0}.src) \wedge \\
					& p_{0}.origin = mail\_client \wedge p_{0}.origin\_body = p_{0}. body \wedge \\
					& (p_{0}.proto = POP3\_REQUEST \vee p_{0}.proto = SMTP\_REQUEST) \wedge \\
					& p_{0}.dst = ip\_mail\_server \wedge p_{0}.emailFrom = mail\_client), \\
					& \forall (n_{0}, p_{0}, t_{0})
				\end{split} \\
				\begin{split}
					\label{mail-client_constr2}
					(re& cv(n_{0}, mail\_client, p_{0}, t_{0})) \implies  (nodeHasAddr(mail\_client, p_{0}.dst)), \\
					& \forall (n_{0}, p_{0}, t_{0})
				\end{split}
			\end{align}
		\end{subequations}
	}%
	\caption{Mail Client model.}
	\label{mail-client_model}
\end{figure}

\paragraph{Web Client model} The web client is based on the end-host model (Formulas~\ref{web-client_constr1} and~\ref{web-client_constr2}), but it can generate only \textit{HTTP\_REQUEST} packets with a web server address as destination (Formula~\ref{web-client_constr1}).
\begin{figure}[h]
	{\footnotesize
		\begin{subequations}
			%WebClient model:
			\begin{align}
				\begin{split}
					\label{web-client_constr1}
					(se& nd(web\_client , n_{0}, p_{0}, t_{0})) \implies (nodeHasAddr(web\_client, p_{0}.src) \wedge \\
					& p_{0}.origin = web\_client \wedge p_{0}.origin\_body = p_{0}.body \wedge \\
					& p_{0}.proto = HTTP\_REQUEST \wedge p_{0}.dst = ip\_web\_server), \\
					& \forall (n_{0}, p_{0}, t_{0})
				\end{split} \\
				\begin{split}
					\label{web-client_constr2}
					(re& cv(n_{0}, web\_client, p_{0}, t_{0})) \implies  (nodeHasAddr(web\_client, p_{0}.dst)), \\
					& \forall (n_{0}, p_{0}, t_{0})
				\end{split}
			\end{align}
		\end{subequations}
	}%
	\caption{Web Client model.}
	\label{web-client_model}
\end{figure}

\paragraph{Web Server model} The web server model is built to send \textit{HTTP\_RESPONSE} packets only if the server has previously received a \textit{HTTP\_REQUEST} packet (Formula~\ref{web-server_constr1}). The sent and received packets must refer to the same \textit{url} field (\textit{$p_{1}.url = p_{0}.url$}):
\begin{figure}[h]
	{\footnotesize
		\begin{subequations}
			%webServer model:
			\begin{align}
				\begin{split}
					\label{web-server_constr1}
					(se& nd(web\_server , n_{0}, p_{0}, t_{0})) \implies (nodeHasAddr(web\_server, p_{0}.src) \wedge \\
					& p_{0}.origin = web\_server \wedge p_{0}.origin\_body = p_{0}.body \wedge \\
					& p_{0}.proto = HTTP\_RESPONSE \wedge \exists (p_{1}, t_{1}) | (t_{1} < t_{0} \\
					& recv(n_{0}, web\_server, p_{1}, t_{1}) \wedge p_{1}.url = p_{0}.url \wedge \\
					& p_{1}.proto = HTTP\_REQUEST \wedge p_{0}.dst = p_{1}.src \wedge p_{0}.src = p_{1}.dst)), \\
					& \forall (n_{0}, p_{0}, t_{0})
				\end{split} \\
				\begin{split}
					\label{web-server_constr2}
					(re& cv( n_{0}, mail\_server, p_{0}, t_{0})) \implies (nodeHasAdrr(web\_server, p_{0}.dst)),\\
					& \forall (n_{0}, p_{0}, t_{0})
				\end{split} 
			\end{align}
		\end{subequations}
	}%
	\caption{Web Server model.}
	\label{web-server_model}
\end{figure}

\paragraph{Anti-Spam model}An anti-spam function was modelled to drop packets from blacklisted mail clients and servers. In fact, the anti-spam behaviour was based on the assumption that each client interested in receiving a new message addressed to it, sends a \texttt{POP3\_REQUEST} to the mail server in order to retrieve the message content. The server, in turn, replies with a \texttt{POP3\_RESPONSE} which contains a special field (\textit{emailFrom}) representing the message sender. The process of sending an email is similarly modelled through SMTP request and response messages. As evident from Formula \ref{anti-spam_constr1}, an anti-spam rejects any message containing a black listed email address (that are set during the creation of the VNF chain model). On the other hand, Formula \ref{anti-spam_constr2} is needed in order to state that a \texttt{POP3\_REQUEST} message is forwarded only after having received it in a previous time instant.
\begin{figure}[h]
	{\footnotesize
		\begin{subequations}
			%Antispam model:
			\begin{align}
				\begin{split}
					\label{anti-spam_constr1}
					(send(anti\_spam, n_{0}, p_{0}, t_{0}) & \wedge p_{0}.protocol = POP3\_RESPONSE) \implies \\
					& \neg isInBlackList(p_{0}.emailFrom) \wedge \exists (n_{1}, t_{1}) \: | \: (t_{1} < t_{0} \\
					& \wedge recv(n_{1}, anti\_spam, p_{0}, t_{1})) \\
					& \forall n_{0}, p_{0}, t_{0}
				\end{split} \\
				\begin{split}
					\label{anti-spam_constr2}
					(send(anti\_spam, n_{0}, p_{0}, t_{0}) & \wedge p_{0}.protocol = POP3\_REQUEST) \implies \\
					& \exists (n_{1}, t_{1}) \: | \: (t_{1} < t_{0} \wedge recv(n_{1}, anti\_spam, p_{0}, t_{1})) \\
					& \forall n_{0}, p_{0}, t_{0}
				\end{split}
			\end{align}
		\end{subequations}
	}%
	\caption{Anti-spam model.}
	\label{anti-spam_model}
\end{figure}
\\
The set of blacklist email addresses is configured through a public function (e.g., \textit{parseConfiguration()}), which gives an interpretation to the \textit{isInBlackList()} uninterpreted function. In particular if the blacklist is empty, VeriGraph will build a constraint like Formula~\ref{anti-spam_configuration1}. Otherwise, let us suppose that the blacklist contains two elements (\textit{BlackList=[mail1, mail2]}), VeriGraph will build the Formula~\ref{anti-spam_configuration2}. In this case, VeriGraph is imposing that the \textit{inInBlackList()} function returns \textit{TRUE} if either \textit{$(emailFrom == mail1)$} or \textit{$(emailFrom == mail2)$} are \textit{TRUE}.
	\begin{figure}[h]
		{\footnotesize
			\begin{subequations}
				%Antispam model:
				\begin{align}
					\begin{split}
						\label{anti-spam_configuration1}
						(isInBlackList(emailFrom) == \;& False), \forall emailFrom
					\end{split} \\
					\begin{split}
						\label{anti-spam_configuration2}
						(isInBlackList(emailFrom) == (& emailFrom == mail1) \; \vee \\
						& (emailFrom == mail2)), \forall emailFrom
					\end{split}
				\end{align}
			\end{subequations}
		}%
	\end{figure}


\paragraph{NAT model}
A different type of function is a NAT, which needs the notion of internal and external networks. This kind of information is modelled by means of a function (\textit{isPrivateAddress}) that checks if an address is registered as private or not. Private addresses are configured when the VNF chain model is initialized. As example of configuration, let us suppose that end-hosts \textit{nodeA} and \textit{nodeB} are internal nodes, hence VeriGraph must add Formula~\ref{nat_configuration1} among its constraint to verify. Here the \textit{isPrivateAddress()} function returns \textit{TRUE} if both \textit{$( address == nodeA\_addr)$} and \textit{$(address == nodeB\_addr)$} are \textit{TRUE}.
\begin{figure}[h]
	{\footnotesize
		\begin{subequations}
			%NAT model:
			\begin{align}
				\begin{split}
					\label{nat_configuration1}
					(isPrivateAddress(address) == (& address == nodeA\_addr \wedge \\
					& address == nodeB\_addr)), \forall address
				\end{split}
			\end{align}
		\end{subequations}
	}%
\end{figure}

In details, the NAT behaviour is modelled by two formulas. Formula~\ref{nat_constr1} states for an internal node which initiates a communication with an external node. In this case, the NAT sends a packet (\textit{$p_0$}) to an external address (\textit{$\neg isPrivateAddress(p_{0}.dst)$}), if and only if it has previously received a packet (\textit{$p_1$}) from an internal node (\textit{$isPrivateAddress(p_{1}.src)$}). The received and sent packets must be equal for all fields, except for the \textit{src}, which must be equal to the NAT public address (\textit{$ip\_nat$}).\\
\begin{figure}[h]
	{\footnotesize
		\begin{subequations}
			%NAT model:
			\begin{align}
			\begin{split}
			\label{nat_constr1}
			(se& nd(nat, n_{0}, p_{0}, t_{0}) \wedge \neg isPrivateAddress(p_{0}.dst)) \implies p_{0}.src = ip\_nat \\
			& \wedge \exists (n_{1}, p_{1}, t_{1}) \: | \: (t_{1} < t_{0} \wedge recv(n_{1}, nat, p_{1}, t_{1}) \wedge isPrivateAddress(p_{1}.src) \\
			& \wedge p_{1}.origin = p_{0}.origin \wedge p_{1}.dst = p_{0}.dst \wedge p_{1}.seq\_no = p_{0}.seq\_no \\
			& \wedge p_{1}.proto = p_{0}.proto \wedge p_{1}.emailFrom = p_{0}.emailFrom \wedge p_{1}.url = p_{0}.url), \\
			& \forall (n_{0}, p_{0}, t_{0})
			\end{split} \\
%			\end{align}
%		\end{subequations}
%	}%
%\end{figure}
%\begin{figure}[t!]
%	{\footnotesize
%		\begin{subequations}
%			%NAT model:
%			\begin{align}
			\begin{split}
			\label{nat_constr2}
			(se& nd(nat, n_{0}, p_{0}, t_{0}) \wedge isPrivateAddress(p_{0}.dst)) \implies \neg isPrivateAddress(p_{0}.src) \\
			& \wedge \exists  (n_{1}, p_{1}, t_{1}) \: | \: (t_{1} < t_{0} \wedge recv(n_{1}, nat, p_{1}, t_{1}) \wedge \neg isPrivateAddress(p_{1}.src) \\
			& \wedge p_{1}.dst = ip\_nat \wedge p_{1}.src = p_{0}.src \wedge p_{1}.origin = p_{0}.origin \\
			& \wedge p_{1}.seq\_no = p_{0}.seq\_no \wedge p_{1}.proto = p_{0}.proto \wedge p_{1}.emailFrom = p_{0}.emailFrom \\
			& \wedge p_{1}.url = p_{0}.url) \wedge \exists (n_{2}, p_{2}, t_{2}) \: | \: (t_{2} < t_{1} \wedge recv(n_{2}, nat, p_{2}, t_{2}) \\
			& \wedge isPrivateAddress(p_{2}.src) \wedge p_{2}.dst = p_{1}.src \wedge p_{2}.dst = p_{0}.src \\
			& \wedge p_{2}.src = p_{0}.dest), \forall (n_{0}, p_{0}, t_{0})
			\end{split}
			\end{align}
		\end{subequations}
	}%
	\caption{NAT model.}
	\label{nat_model}
\end{figure}
On the other hand, the traffic from the external network to the private is modelled by Formula ~\ref{nat_constr2}. In this case, if the NAT is sending a packet to an internal address (\textit{$isPrivateAddress(p_{0}.dst)$}), this packet (\textit{$p_0$}) must have an external address as its source (\textit{$\neg isPrivateAddress(p_{0}.src)$}). Moreover, \textit{$p_0$} must be preceded by another packet (\textit{$p_1$}), which is, in turn, received by the NAT and it is equal to \textit{$p_0$} for all the other fields. It is worth noting that, generally, a communication between internal and external nodes cannot be started by the external node in presence of a NAT. As a consequence, this condition is expressed in the Formula~\ref{nat_constr2} by imposing that \textit{$p_1$} must be preceded by another packet \textit{$p_2$} (\textit{$(t_{2} < t_{1} \wedge recv(n_{2}, nat, p_{2}, t_{2})$}), sent to the NAT from an internal node (\textit{$isPrivateAddress(p_{2}.src)$}).

\paragraph{Web Cache model} A simple version of web cache can be modelled with three formulas (Formulas~\ref{cache_constr1},~\ref{cache_constr2} and~\ref{cache_constr3}), where we have a notion of internal addresses (\textit{$isInternal$} function), which are configured when the chain model is created. VeriGraph follows a similar approach to the NAT model (Formula~\ref{nat_configuration1}) for configuring the internal nodes. For instance, if the internal network is composed of two nodes \textit{nodeA} and \textit{nodeB}, VeriGraph will give an interpretation to the \textit{isInternal} function by means of Formula~\ref{web-cache_configuration1}.
\begin{figure}[h]
	{\footnotesize
		\begin{subequations}
			%Web cache model:
			\begin{align}
				\begin{split}
					\label{web-cache_configuration1}
					(isInternal(node) == (& node == nodeA \wedge \\
					& node == nodeB)), \forall node
				\end{split}
			\end{align}
		\end{subequations}
	}%
\end{figure}

This model was designed to work with web end-hosts (i.e., web client and server). In details, formula~\ref{cache_constr1} states that: a packet sent from the cache to a node belonging to the external network (\textit{$\neg isInternal(n_{0})$}), implies a previous HTTP request packet (\textit{$p_{0}.proto = HTTP\_REQ$}) and received from an internal node, which cannot be served by the cache (\textit{$\neg isInCache(p_{0}.url, t_{0})$}), otherwise the request would have not been forwarded towards the external network. 

\begin{figure}[h]
	{\footnotesize
		\begin{subequations}
			%Web cache model:
			\begin{align}
			\begin{split}
			\label{cache_constr1}
			(send(cache&, n_{0}, p_{0}, t_{0}) \wedge \neg isInternal(n_{0})) \implies \neg isInCache(p_{0}.url, t_{0}) \\
			& \wedge p_{0}.proto = HTTP\_REQ \wedge \exists (t_{1}, n_{1}) \: | \: (t_{1} < t_{0} \\
			& \wedge isInternalNode(n_{1}) \wedge recv(n_{1}, cache, p_{0}, t_{1})), \\
			& \forall (n_{0}, p_{0}, t_{0})
			\end{split} \\
			\begin{split}
			\label{cache_constr2}
			(send(cache&, n_{0}, p_{0}, t_{0}) \wedge isInternal(n_{0})) \implies isInCache(p_{0}.url, t_{0}) \\
			& \wedge p_{0}.proto = HTTP\_RESP \wedge p_{0}.src = p_{1}.dst \wedge p_{0}.dst = p_{1}.src \wedge \\
			& \wedge \exists  (p_{1}, t_{1}) \: | \: (t_{1} < t_{0} \wedge p_{1}.proto = HTTP\_REQ  \\
			& \wedge p_{1}.url = p_{0}.url \wedge recv(n_{0}, cache, p_{1}, t_{1})), \\
			& \forall (n_{0}, p_{0}, t_{0})
			\end{split} \\
			\begin{split}
			\label{cache_constr3}
			isInCache&(u_{0}, t_{0}) \implies \exists (t_{1}, t_{2}, p_{1}, p_{2}, n_{1}, n_{1}) \: | \: (t_{1} < t_{2} \wedge t_{1} < t_{0} \wedge t_{2} < t_{0} \\
			& \wedge recv(n_{1}, cache, p_{1}, t_{1}) \wedge recv(n_{2}, cache, p_{2}, t_{2}) \wedge p_{1}.proto = HTTP\_REQ \\
			& \wedge p_{1}.url = u_{0} \wedge p_{2}.proto = HTTP\_RESP \wedge p_{2}.url = u_{0} \wedge isInternal(n_{2})) \\
			& \forall (u_{0}, t_{0})
			\end{split}
			\end{align}
			\end{subequations}
			}%
			\caption{Web cache model.}
			\label{cache_model}
\end{figure}

Formula~\ref{cache_constr2} states that a packet sent from the cache to the internal network contains a \texttt{HTTP\_RESPONSE} for an URL which was in cache when the request has been received. We also state that the packet received from the internal network is a \texttt{HTTP\_REQUEST} and the target URL is the same as the response (\textit{$p_{1}.protocol = HTTP\_REQ \wedge p_{1}.url = p_{0}.url$}). 

The final formula (Formula~\ref{cache_constr3}) expresses a constraint that the \textit{isInCache()} function must respect. In particular, we state that a given URL ($u_{0}$) is in cache at time $t_{0}$ if (and only if) a request packet was received at time $t_{1}$ (where $t_{1} < t_{0}$) for that URL and a subsequent packet was received at time $t_{2}$ (where $t_{2} < t_{0} \wedge t_{2} > t_{1}$) carrying the corresponding \texttt{HTTP\_RESPONSE}.

\paragraph{ACL firewall model} An ACL firewall is a simple firewall that drops packets based on its internal Access Control List (ACL), configured when the chain model is initialized. In particular the ACL list is managed through the uninterpreted function \textit{$acl\_func()$}. A possible interpretation is given by VeriGraph though the Formula~\ref{acl-firewall_configuration1}, when the ACL list contains two entries, like for example \textit{$ACL = [<src_1, dst_1>,<src_2,dst_2>]$}.
	\begin{figure}[h]
		{\footnotesize
			\begin{subequations}
				%Antispam model:
				\begin{align}
					\begin{split}
						\label{acl-firewall_configuration1}
						(acl\_func(a, b) == \;& ((a == src_1 \wedge b == dst_1 ) \; \vee \\
						& (a == src_2 \wedge b == dst_2))), \forall a, b
					\end{split}
				\end{align}
			\end{subequations}
		}%
	\end{figure}


Hence, if an ACL firewall sends a packet, this implies that the firewall has previously received a packet of which the source and destination address are not contained in the ACL list.
\begin{figure}[h]
	{\footnotesize
		\begin{subequations}
			\begin{align}
				\begin{split}
					\label{acl-fw_constr1}
					(send(fw, n_{0}, p_{0}, t_{0})) \implies (&\exists (t_{1}, n_{1}) | t_{1} < t_{0} \wedge recv(n_{1}, fw, p_{0}, t_{1}) \wedge \\
					&\neg acl\_func(p_{0}.src, p_{0}.dst)), \\
					& \forall (n_{0}, p_{0}, t_{0})
				\end{split}
			\end{align}
		\end{subequations}
	}%
	\caption{ACL Firewall model}
	\label{acl-fw_model}
\end{figure}


\paragraph{Learning firewall model} The learning firewall behaviour follows the fundamental principals of network forwarding: if the firewall is sending a packet, this implies that the function has previously received that packet (Formula~\ref{learning-fw_constr1}). However the forwarding decisions of a learning firewall depend on an internal Access Control List: each ACL entry specifies a pair of addresses allowed to exchange packets and are configured through the same mechanism defined for the ACL firewall model (Formula~\ref{acl-firewall_configuration1}). The ACL entries define also the direction of the allowed traffic flows, which is, for instance, if the ACL list contains an entry like \textit{$<nodeA\_addr, nodeB\_addr>$}, the learning firewall will forward packet from node \textit{A} to \textit{B}, but no the vice-versa. The traffic in the opposite direction (i.e., from \textit{B} to \textit{A}) is allowed if and only if node \textit{A} has previously opened a connection toward \textit{B} (that is \textit{A} has sent a packet to \textit{B} - Formula~\ref{learning-fw_constr2}):
\begin{figure}[h]
	{\footnotesize
		\begin{subequations}
			%Antispam model:
			\begin{align}
				\begin{split}
					\label{learning-fw_constr1}
					(send(fw, n_{0}, p_{0}, t_{0})) \implies (&\exists (t_{1}, n_{1}) | \; t_{1} < t_{0} \wedge recv(n_{1}, fw, p_{0}, t_{1})),\forall (n_{0}, p_{0}, t_{0})
				\end{split}
			\end{align}
			%	\end{subequations}
			%	\begin{subequations}
			\begin{align}
				\begin{split}
					\label{learning-fw_constr2}
					(send(fw, n_{0}, & p_{0}, t_{0})  \wedge \neg acl\_func(p_{0}.src, p_{0}.dst)) \implies (\exists ( n_{1},t_{1}) \; | \; send(fw, n_{1}, p_{1},t_{1}) \wedge \\
					& t_{1} + 1 < t_{0} \wedge acl\_func(p_{1}.src, p_{1}.dst) \wedge  p_{0}.src = p_{1}.dst \wedge  p_{0}.dst = p_{1}.src \wedge \\
					& p_{0}.src\_port = p_{1}.dst\_port \wedge p_{0}.dst\_port = p_{1}.src\_port), \\
					& \forall (n_{0}, p_{0}, t_{0})
				\end{split}
			\end{align}
		\end{subequations}
	}%
	\caption{Learning Firewall model}
	\label{learning-fw_model}
\end{figure}


