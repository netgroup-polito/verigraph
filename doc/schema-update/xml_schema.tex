\section{Verigraph XML Schema Documentation}

\subsection{Differences with the previous version}

In this work the original XML schema was improved in several ways. Here we list the changes and improvements that were made:

\begin{table}[htp]
\centering
\resizebox{\textwidth}{!}{%
\begin{tabular}{lp{15cm}}
\toprule
\textbf{Improvements} & \textbf{Details}               \\ \midrule                                   
New constraints             &  The original XML schema did not include most of the reference constraints required to check the existence of a node that was referred by another node. For instance, a NAT or a Firewall could be targetting a node that did not exist, and a vpnaccess element could be referring to a node that was not a vpnexit element. \\ \midrule
PacketType               &  In the original XML schema there were two elements (Endhost and FieldModifier) that used the same attributes to describe a subset of the fields that composed a packet. In the new version, this redundancy has been removed by introducing a new complex type PacketType that centralizes the description of the subset of the supported fields. Moreover, the same complex type can also be used to characterize a traffic flow. \\ \midrule
Policies, Policy, Restrictions               &  New elements and types have been introduced to model the different policies that can be applied to a graph.  \\ \midrule
Firewall                    &  The firewall's ACL has been extended to include also the source/destination ports and the transport protocol.
 \\ \bottomrule
\end{tabular}%
}
\end{table}

\subsection{Verigraph XML Schema}

\begin{lstlisting}[language=XML, caption=XML Example]
<?xml version="1.0" encoding="UTF-8"?>
<graphs xmlns:xsi="http://www.w3.org/2001/XMLSchema-instance" xsi:noNamespaceSchemaLocation="xml_components_v2.xsd">
  <graph id="0">
    <node functional_type="FIREWALL" id="0" name="fw">
      <neighbour id="0" name="nodeSrc"/>
      <neighbour id="1" name="nodeDest"/>
      <configuration description="Simple Description" id="0" name="conf1">
        <firewall>
          <elements>
            <source>nodeSrc</source>
            <destination>nodeDest</destination>
            <srcPort>30</srcPort>
            <destPort>30</destPort>
            <protocol>TCP</protocol>
          </elements>
        </firewall>
      </configuration>
    </node>
    <node functional_type="ENDPOINT" id="1" name="nodeSrc">
      <neighbour id="0" name="fw"/>
      <configuration description="Simple description" id="0" name="conf2">
        <endhost />
      </configuration>
    </node>
    <node functional_type="ENDPOINT" id="2" name="nodeDest">
      <neighbour id="0" name="fw"/>
      <configuration description="Simple description" id="0" name="conf3">
        <endhost />
      </configuration>
    </node>

    <policies>
      <policy destination="nodeDest" id="0" name="policyExample" source="nodeSrc">
        <trafficflow protocol="HTTP_REQUEST"/>
        <restrictions type="SELECTION">
          <genericFunction type="FIREWALL"/>
        </restrictions>
      </policy>
    </policies>
  </graph>
</graphs>
\end{lstlisting}

\subsubsection*{Graphs}
Graphs is the root element of the XML schema. It contains a list of \textbf{Graph} elements.

\subsubsection*{Graph}
A Graph is a set of services that will be deployed in the network. It is contained inside a list of Graphs.
Graph is characterised by
\begin{itemize}
 \item A unique \textbf{ID}
 \item A list of \textbf{Node} elements
 \item (Optionally) A \textbf{Policies} element that contains a list of \textbf{Policy}
\end{itemize}

\begin{warning}
\textbf{Warning}: You must define a Graph that has at least one Node.
\end{warning}

\begin{lstlisting}[language=XML, caption=Graphs Example]
<graphs xmlns:xsi="http://www.w3.org/2001/XMLSchema-instance" xsi:noNamespaceSchemaLocation="xml_components_v2.xsd">
  <graph id="0">
    <node functional_type="FIREWALL" id="0" name="fw">
        ...
    </node>
    <node functional_type="ENDPOINT" id="1" name="nodeSrc">
        ...
    </node>
    <node functional_type="ENDPOINT" id="2" name="nodeDest">
        ...
    </node>

    <policies>
      <policy destination="nodeDest" id="0" source="nodeSrc">
        ...
      </policy>
    </policies>
  </graph>
</graphs>
\end{lstlisting}

\begin{lstlisting}[language=XML, caption=Graphs schema code snippet]
<xsd:element name="graphs">
    <xsd:complexType>
        <xsd:sequence>
            <xsd:element ref="graph" maxOccurs="unbounded"
                minOccurs="0" />
        </xsd:sequence>
    </xsd:complexType>
    <xsd:unique name="uniqueGraph">
        <xsd:selector xpath="graph" />
        <xsd:field xpath="@id" />
    </xsd:unique>
</xsd:element>
<xsd:element name="graph">
    <xsd:complexType>
        <xsd:sequence>
            <xsd:element ref="node" maxOccurs="unbounded"/>
            <xsd:element ref="policies" minOccurs="0"/>
        </xsd:sequence>
        <xsd:attribute name="id" type="xsd:long" use="optional" />
    </xsd:complexType>
    ...
</xsd:element>
\end{lstlisting}

\subsubsection*{Node}
A Node is a logical network element that correspond to a Network Function. A node is characterised
by:
\begin{itemize}
 \item A unique \textbf{Name}
 \item A \textbf{Functional Type}
 \item A list of \textbf{Neighbour Node Names}
 \item A \textbf{Configuration} for the Functional Type
\end{itemize}


\begin{warning}
\textbf{Warning}: Pay attention when you define the neighbours of a node. The neighbour node names must exist within the specific graph.
\end{warning}

\begin{lstlisting}[language=XML, caption=Node Example]
<node functional_type="ENDPOINT" id="1" name="nodeSrc">
  <neighbour id="0" name="fw"/>
  <configuration ...>
    ...
  </configuration>
</node>
\end{lstlisting}

\begin{lstlisting}[language=XML, caption=Node schema code snippet]
<xsd:element name="graph">
    ...
    <xsd:unique name="uniqueNodeId">
            <xsd:selector xpath="node" />
            <xsd:field xpath="@id" />
        </xsd:unique>
    ...
</xsd:element>
<xsd:element name="node">
    <xsd:complexType>
        <xsd:sequence>
            <xsd:element ref="neighbour" maxOccurs="unbounded"
                minOccurs="0" />
            <xsd:element ref="configuration" maxOccurs="1"
                minOccurs="1" />
        </xsd:sequence>
        <xsd:attribute name="id" type="xsd:long" use="optional" />
        <xsd:attribute name="name" type="xsd:string"
            use="required" />
        <xsd:attribute name="functional_type" type="functionalTypes"
            use="required" />
    </xsd:complexType>
    ...
</xsd:element>
\end{lstlisting}


\subsubsection*{Functional Type}

A Node can be a:
\begin{itemize}
 \item \textbf{FIREWALL}
 \item \textbf{ENDHOST}
 \item \textbf{ENDPOINT}
 \item \textbf{ANTISPAM}
 \item \textbf{CACHE}
 \item \textbf{DPI}
 \item \textbf{MAILCLIENT}
 \item \textbf{MAILSERVER}
 \item \textbf{NAT}
 \item \textbf{VPNACCESS}
 \item \textbf{VPNEXIT}
 \item \textbf{WEBCLIENT}
 \item \textbf{WEBSERVER}
 \item \textbf{FIELDMODIFIER}
\end{itemize}

\subsubsection*{Configuration}
In this section we describe the different type of configurations that can be provided. A configuration is characterized by an \emph{unique} name and by an \emph{optional} description.

\paragraph{Firewall}
A Firewall Configuration contains a list of ACLs (elements). The ACL defines a tuple of:
\begin{itemize}
 \item Source node ID
 \item Destination node ID
 \item Source port
 \item Destination port
 \item The \textbf{Transport Protocol Type}
\end{itemize}
This tuple represents the connection that will be blocked.

The \textbf{Transport Protocol Type} can be:
\begin{itemize}
 \item \textbf{TCP}
 \item \textbf{UDP}
\end{itemize}

\begin{warning}
\textbf{Warning}: Source and destination nodes must exist in the same graph.
\end{warning}

\begin{lstlisting}[language=XML, caption=Firewall Configuration Example]
<configuration description="Simple Description" id="0" name="conf1">
    <firewall>
      <elements>
        <source>nodeSrc</source>
        <destination>nodeDest</destination>
        <srcPort>30</srcPort>
        <destPort>30</destPort>
        <protocol>TCP</protocol>
      </elements>
    </firewall>
</configuration>
\end{lstlisting}

\begin{lstlisting}[language=XML, caption=Firewall schema code snippet]
<xsd:element name="graph">
    ...
    <xsd:keyref name="keyRefNodeSrc" refer="keyNode">
            <xsd:selector xpath="node/neighbour" />
            <xsd:field xpath="@name" />
    </xsd:keyref>
    <!-- Firewall -->
    <xsd:keyref name="keyRefFirewallSrc" refer="keyNode">
        <xsd:selector xpath="node/configuration/firewall/elements" />
        <xsd:field xpath="source" />
    </xsd:keyref>
    <xsd:keyref name="keyRefFirewallDest" refer="keyNode">
        <xsd:selector xpath="node/configuration/firewall/elements" />
        <xsd:field xpath="destination" />
    </xsd:keyref>
    ...
</xsd:element>
<xsd:element name="firewall">
    <xsd:complexType>
        <xsd:sequence>
            <xsd:element ref="elements" minOccurs="0" maxOccurs="unbounded" />
        </xsd:sequence>
    </xsd:complexType>
</xsd:element>
<xsd:element name="elements">
    <xsd:complexType>
        <xsd:sequence>
            <xsd:element name="source" type="xsd:string" />
            <xsd:element name="destination" type="xsd:string" />
            <xsd:element name="srcPort" type="xsd:positiveInteger" />
            <xsd:element name="destPort" type="xsd:positiveInteger" />
            <xsd:element name="protocol" type="transportProtocolTypes" />
        </xsd:sequence>
    </xsd:complexType>
</xsd:element>
\end{lstlisting}

\paragraph{EndHost} 

An EndHost Configuration contains a description of the traffic flow that is generated from/received by this node.

\begin{lstlisting}[language=XML, caption=EndHost Configuration Example]
<configuration description="Simple Description" id="0" name="conf1">
    <endhost body="lorem ipsum"/>
</configuration>
\end{lstlisting}

\begin{lstlisting}[language=XML, caption=EndHost schema code snippet]
<xsd:element name="endhost" type="packetType" />
\end{lstlisting}

\paragraph{FieldModifier}

A Field Modifier Configuration contains the fields of the traffic flow that are modified by this node.

\begin{lstlisting}[language=XML, caption=FieldModifier Configuration Example]
<configuration description="Simple Description" id="0" name="conf1">
    <fieldmodifier body="This will be changed"/>
</configuration>
\end{lstlisting}

\begin{lstlisting}[language=XML, caption=Fieldmodifier schema code snippet]
<xsd:element name="fieldmodifier" type="packetType"/>
\end{lstlisting}

\paragraph{Packet Type}

Each packet and traffic flow can be characterized by the following attributes (all of them are optional):
\begin{itemize}
 \item Body
 \item Sequence
 \item Protocol
 \item Email From
 \item URL
 \item Options
 \item Destination
\end{itemize}
\begin{lstlisting}[language=XML, caption=PacketType schema code snippet]
<xsd:complexType name="packetType">
    <xsd:attribute name="body" type="xsd:string" />
    <xsd:attribute name="sequence" type="xsd:integer" />
    <xsd:attribute name="protocol" type="protocolTypes" />
    <xsd:attribute name="email_from" type="xsd:string" />
    <xsd:attribute name="url" type="xsd:string" />
    <xsd:attribute name="options" type="xsd:string" />
	<xsd:attribute name="destination" type="xsd:string" />
</xsd:complexType>
\end{lstlisting}

\paragraph{AntiSpam}

An Antispam Configuration contains a list of source nodes that represent the blacklisted mail clients and servers.

\begin{warning}
\textbf{Warning}: The nodes to which the Antispam refers must exist in the same graph.
\end{warning}

\begin{lstlisting}[language=XML, caption=Antispam Configuration Example]
<configuration description="Simple Description" id="0" name="conf1">
    <antispam>
    	<source name="node1" />
    </antispam>
</configuration>
\end{lstlisting}

\begin{lstlisting}[language=XML, caption=Antispam schema code snippet]
<xsd:element name="graph">
    ...
    <xsd:keyref name="keyRefNodeSrc" refer="keyNode">
            <xsd:selector xpath="node/neighbour" />
            <xsd:field xpath="@name" />
    </xsd:keyref>
    ...
    <!-- Antispam  -->
    <xsd:keyref name="keyRefAntispam" refer="keyNode">
        <xsd:selector xpath="node/configuration/antispam/source" />
        <xsd:field xpath="@name" />
    </xsd:keyref>
    ...
</xsd:element>
<xsd:element name="antispam">
    <xsd:complexType>
        <xsd:sequence minOccurs="1" maxOccurs="unbounded">
            <xsd:element name="source" type="addressType" />
        </xsd:sequence>
    </xsd:complexType>
</xsd:element>
\end{lstlisting}

\paragraph{Cache}

A Cache Configuration contains a list of resources. A resource is a node internal to the cache, and it is characterized by the name of the node it refers to.

\begin{warning}
\textbf{Warning}: The nodes to which the cache refers must exist in the same graph.
\end{warning}

\begin{lstlisting}[language=XML, caption=Cache Configuration Example]
<configuration description="Simple Description" id="0" name="conf1">
    <cache>
    	<resource name="node1"/>
    	<resource name="node2"/>
    </cache>
</configuration>
\end{lstlisting}

\begin{lstlisting}[language=XML, caption=Cache schema code snippet]
<xsd:element name="graph">
    ...
    <xsd:keyref name="keyRefNodeSrc" refer="keyNode">
            <xsd:selector xpath="node/neighbour" />
            <xsd:field xpath="@name" />
    </xsd:keyref>
    ...
    <!-- Cache -->
    <xsd:keyref name="keyRefCache" refer="keyNode">
        <xsd:selector xpath="node/configuration/cache/resource" />
        <xsd:field xpath="@name" />
    </xsd:keyref>
    ...
</xsd:element>
<xsd:element name="cache">
    <xsd:complexType>
        <xsd:sequence minOccurs="1" maxOccurs="unbounded">
            <xsd:element name="resource" type="addressType" />
        </xsd:sequence>
    </xsd:complexType>
</xsd:element>
\end{lstlisting}

\paragraph{DPI}

A DPI Configuration contains a list of notAllowed elements, that defines the strings that can’t be present inside a packet otherwise it will be dropped.

\begin{lstlisting}[language=XML, caption=DPI Configuration Example]
<configuration description="Simple Description" id="0" name="conf1">
    <dpi>
        <notAllowed>Some String</notAllowed>
    </dpi>
</configuration>
\end{lstlisting}

\begin{lstlisting}[language=XML, caption=DPI schema code snippet]
<xsd:element name="dpi">
    <xsd:complexType>
        <xsd:sequence minOccurs="1" maxOccurs="unbounded">
            <xsd:element name="notAllowed" type="xsd:string" />
        </xsd:sequence>
    </xsd:complexType>
</xsd:element>
\end{lstlisting}

\paragraph{MailClient}

A Mail Client Configuration contains the Mail Server name.

\begin{warning}
\textbf{Warning}: The Mail Server to which this node refers must exist within the specific graph.
\end{warning}

\begin{lstlisting}[language=XML, caption=MailClient Configuration Example]
<configuration description="Simple Description" id="0" name="conf1">
    <mailclient mailserver="ServerName"/>
</configuration>
\end{lstlisting}

\paragraph{MailServer}

A Mail Server Configuration contains the Mail Server name.

\begin{lstlisting}[language=XML, caption=MailServer Configuration Example]
<configuration description="Simple Description" id="0" name="conf1">
    <mailserver>
    	<name>ServerName</name>
    </mailserver>
</configuration>
\end{lstlisting}

\begin{lstlisting}[language=XML, caption=MailClient and MailServer schema code snippet]
<xsd:element name="graph">
    ...
    <!-- WebClient / WebServer -->
    <xsd:key name="keyWebServer">
        <xsd:selector xpath="node/configuration/webserver" />
        <xsd:field xpath="name" />
    </xsd:key>
    <xsd:keyref name="keyRefWebClient" refer="keyWebServer">
        <xsd:selector xpath="node/configuration/webclient" />
        <xsd:field xpath="@nameWebServer" />
    </xsd:keyref>
    <!-- WebClient / WebServer -->
    <xsd:key name="keyMailServer">
        <xsd:selector xpath="node/configuration/mailserver" />
        <xsd:field xpath="name" />
    </xsd:key>
    <xsd:keyref name="keyRefMailClient" refer="keyMailServer">
        <xsd:selector xpath="node/configuration/mailclient" />
        <xsd:field xpath="@mailserver" />
    </xsd:keyref>
    ...
</xsd:element>
<xsd:element name="mailclient">
    <xsd:complexType>
        <xsd:attribute name="mailserver" type="xsd:string"
            use="required" />
    </xsd:complexType>
</xsd:element>
<xsd:element name="mailserver">
    <xsd:complexType>
        <xsd:sequence>
            <xsd:element name="name" type="xsd:string" />
        </xsd:sequence>
    </xsd:complexType>
</xsd:element>
\end{lstlisting}

\paragraph{NAT}
A NAT Configuration contains a list of internal nodes. The \textbf{source} element define a node in the graph.

\begin{warning}
\textbf{Warning}: The nodes to which the NAT refers must exist in the same graph.
\end{warning}

\begin{lstlisting}[language=XML, caption=NAT Configuration Example]
<configuration description="Simple Description" id="0" name="conf1">
    <nat>
    	<source name="node1"/>
    	<source name="node2"/>
    </nat>
</configuration>
\end{lstlisting}

\begin{lstlisting}[language=XML, caption=NAT schema code snippet]
<xsd:element name="graph">
    ...
    <xsd:keyref name="keyRefNodeSrc" refer="keyNode">
            <xsd:selector xpath="node/neighbour" />
            <xsd:field xpath="@name" />
    </xsd:keyref>
    ...
    <!-- NAT -->
    <xsd:keyref name="keyRefNAT" refer="keyNode">
        <xsd:selector xpath="node/configuration/nat/source" />
        <xsd:field xpath="@name" />
    </xsd:keyref>
    ...
</xsd:element>
<xsd:element name="nat">
    <xsd:complexType>
        <xsd:sequence minOccurs="1" maxOccurs="unbounded">
            <xsd:element name="source" type="addressType" />
        </xsd:sequence>
    </xsd:complexType>
</xsd:element>
\end{lstlisting}

\paragraph{VPNAccess}

A VpnAccess Configuration contains the VpnExit name.

\begin{warning}
\textbf{Warning}: The VPNExit name must exist within the same graph.
\end{warning}

\begin{lstlisting}[language=XML, caption=VPNAccess Configuration Example]
<configuration description="Simple Description" id="0" name="conf1">
    <vpnaccess vpnexit="vpe1">
        <name>vpa1</name>
    </vpnaccess>
</configuration>
\end{lstlisting}

\paragraph{VPNExit}

A VpnExit Configuration contains the VpnAccess name.

\begin{warning}
\textbf{Warning}: The VPNAccess name must exist within the same graph.
\end{warning}

\begin{lstlisting}[language=XML, caption=VPNExit Configuration Example]
<configuration description="Simple Description" id="0" name="conf1">
    <vpnexit vpnaccess="vpa1">
    	<name>vpe1</name>
    </vpnexit>
</configuration>
\end{lstlisting}

\begin{lstlisting}[language=XML, caption=VPNAccess and VPNExit schema code snippet]
<xsd:element name="graph">
    ...
    <!-- VPN -->
    <xsd:key name="keyVPNAccess">
        <xsd:selector xpath="node/configuration/vpnaccess" />
        <xsd:field xpath="name" />
    </xsd:key>
    <xsd:keyref name="keyRefVPNAccess" refer="keyVPNAccess">
        <xsd:selector xpath="node/configuration/vpnexit" />
        <xsd:field xpath="@vpnaccess" />
    </xsd:keyref>
    <xsd:key name="keyVPNExit">
        <xsd:selector xpath="node/configuration/vpnexit" />
        <xsd:field xpath="name" />
    </xsd:key>
    <xsd:keyref name="keyRefVPNExit" refer="keyVPNExit">
        <xsd:selector xpath="node/configuration/vpnaccess" />
        <xsd:field xpath="@vpnexit" />
    </xsd:keyref>
    ...
</xsd:element>
<xsd:element name="vpnaccess">
    <xsd:complexType>
    	<xsd:sequence>
            <xsd:element name="name" type="xsd:string" />
        </xsd:sequence>
        <xsd:attribute name="vpnexit" type="xsd:string"
            use="required" />
    </xsd:complexType>
</xsd:element>
<xsd:element name="vpnexit">
    <xsd:complexType>
    	<xsd:sequence>
            <xsd:element name="name" type="xsd:string" />
        </xsd:sequence>
        <xsd:attribute name="vpnaccess" type="xsd:string"
            use="required" />
    </xsd:complexType>
</xsd:element>
\end{lstlisting}

\paragraph{WebClient}

A Web Client Configuration contains the Web Server name.

\begin{warning}
\textbf{Warning}: The Web Server to which this node refers must exist within the specific graph.
\end{warning}

\begin{lstlisting}[language=XML, caption=WebClient Configuration Example]
<configuration description="Simple Description" id="0" name="conf1">
    <webclient nameWebServer="ServerName"/>
</configuration>
\end{lstlisting}

\paragraph{WebServer}

A Web Server Configuration contains the Web Server name.

\begin{lstlisting}[language=XML, caption=WebServer Configuration Example]
<configuration description="Simple Description" id="0" name="conf1">
    <webserver>
    	<name>ServerName</name>
    </webserver>
</configuration>
\end{lstlisting}
\begin{lstlisting}[language=XML, caption=WebClient and WebServer schema code snippet]
<xsd:element name="graph">
    ...
    <!-- WebClient / WebServer -->
    <xsd:key name="keyWebServer">
        <xsd:selector xpath="node/configuration/webserver" />
        <xsd:field xpath="name" />
    </xsd:key>
    <xsd:keyref name="keyRefWebClient" refer="keyWebServer">
        <xsd:selector xpath="node/configuration/webclient" />
        <xsd:field xpath="@nameWebServer" />
    </xsd:keyref>
    <xsd:key name="keyMailServer">
        <xsd:selector xpath="node/configuration/mailserver" />
        <xsd:field xpath="name" />
    </xsd:key>
    <xsd:keyref name="keyRefMailClient" refer="keyMailServer">
        <xsd:selector xpath="node/configuration/mailclient" />
        <xsd:field xpath="@mailserver" />
    </xsd:keyref>
    ...
</xsd:element>
<xsd:element name="webclient">
    <xsd:complexType>
        <xsd:attribute name="nameWebServer" type="xsd:string"
            use="required" />
    </xsd:complexType>
</xsd:element>
<xsd:element name="webserver">
    <xsd:complexType>
        <xsd:sequence>
            <xsd:element name="name" type="xsd:string" />
        </xsd:sequence>
    </xsd:complexType>
</xsd:element>
\end{lstlisting}

\subsubsection*{Policy}
A Policy contains a reachability policy that we want to verify in a network graph. It is contained inside a list of Policies.

A Policy is characterised by
\begin{itemize}
 \item A unique \textbf{ID}
 \item The \textbf{Source Node} name
 \item The \textbf{Destination Node} name
 \item The \textbf{Traffic Flow} to which this policy refers.
 \item A \textbf{Restrictions} element containing the restrictions to be checked.
\end{itemize}

\begin{warning}
\textbf{Warning}: The source and destination names must exist within the same graph of the policy.
\end{warning}

\begin{lstlisting}[language=XML, caption=Policies Example]
<policies>
	<policy destination="node2" source="node1" id="1"  name="example">
		<trafficflow body="Lorem Ipsum"/>
		<restrictions ...>
	      ...
	    </restrictions>
	</policy>
</policies>
\end{lstlisting}

\begin{lstlisting}[language=XML, caption=Policy schema code snippet]
<xsd:element name="graph">
    ...
    <!-- Integrity constraints on the policies that must refer to existing nodes -->
    <xsd:keyref name="keyRefPolicySrc" refer="keyNode">
        <xsd:selector xpath="policies/policy" />
        <xsd:field xpath="@source" />
    </xsd:keyref>
    <xsd:keyref name="keyRefPolicyDest" refer="keyNode">
        <xsd:selector xpath="policies/policy" />
        <xsd:field xpath="@destination" />
    </xsd:keyref>
    <xsd:keyref name="keyRefPolicyFunction" refer="keyNode">
        <xsd:selector xpath="policies/policy/restrictions/exactFunction" />
        <xsd:field xpath="name" />
    </xsd:keyref>
    ...
</xsd:element>
<xsd:element name="policies">
     	<xsd:complexType>
            <xsd:sequence>
                <xsd:element ref="policy" maxOccurs="unbounded"/>
            </xsd:sequence>
        </xsd:complexType>
        <xsd:unique name="uniquePolicy">
            <xsd:selector xpath="policy" />
            <xsd:field xpath="@id" />
        </xsd:unique>
    </xsd:element>
    <xsd:element name="policy">
        <xsd:complexType>
            <xsd:sequence>
                <xsd:element name="trafficflow" type="packetType"/>
                <xsd:element ref="restrictions"/>
            </xsd:sequence>
            <xsd:attribute name="id" type="xsd:long"
                use="required" />
            <xsd:attribute name="name" type="xsd:string"
                use="required" />
            <xsd:attribute name="source" type="xsd:string"
                use="required" />
            <xsd:attribute name="destination" type="xsd:string"
                use="required" />
        </xsd:complexType>
</xsd:element>
\end{lstlisting}

\subsubsection*{Restrictions}
A Restrictions element is composed of a type and a set of service functions which the specified traffic must pass through.

The possible types of a Restrictions element can be:
\begin{itemize}
 \item \textbf{SELECTION}
 \item \textbf{SET}
 \item \textbf{SEQUENCE}
 \item \textbf{LIST}
\end{itemize}

The service functions can be of two types:
\begin{itemize}
 \item \textbf{genericFunction}: It describes a generic service function, without specifying the name of the exact node to which it refers.
 \item \textbf{exactFunction}: It describes a specific service function that \textbf{must exist} within the same graph of the policy.
\end{itemize}

\begin{lstlisting}[language=XML, caption=Restrictions Example]
<restrictions type="SELECTION">
    <genericFunction type="FIREWALL" />
    <exactFunction>
  	 <name>node1</name>
    </exactFunction>
    <genericFunction type="NAT" />
</restrictions>
\end{lstlisting}

\begin{lstlisting}[language=XML, caption=Restrictions schema code snippet]
<xsd:element name="restrictions">
    <xsd:complexType>
    	<xsd:choice maxOccurs="unbounded">
    		<xsd:element ref="genericFunction"/>
    		<xsd:element ref="exactFunction" />
    	</xsd:choice>
        <xsd:attribute name="type" type="restrictionTypes"
            use="required" />
    </xsd:complexType>
</xsd:element>
<xsd:element name="genericFunction">
	<xsd:complexType>
       <xsd:attribute name="type" type="functionalTypes" use="required"/>
   	</xsd:complexType>
</xsd:element>
<xsd:element name="exactFunction">
	<xsd:complexType>
       <xsd:sequence>
           <xsd:element name="name" type="xsd:string" />
       </xsd:sequence>
   	</xsd:complexType>
</xsd:element>
\end{lstlisting}